%_% These lines starting with %_% are used as annotations for GNU source-highlight package to be able to selectively
%_% convert parts of this file into markup. These can be removed before using this file.
%_% Try: grep -v '^%_%' cheatsheet.tex
%_% Start HEADER
\documentclass[a4paper,extrafontsizes,12pt,twoside,openany]{memoir}
\chapterstyle{bianchi}
\aliaspagestyle{chapter}{empty}
\pagestyle{empty}
%_% End HEADER

%_% PARAGRAPHSTYLES
\setlength\parindent{0in}
\setlength\parskip{1ex}
%_% PARAGRAPHSTYLES

%_% XETEXPACKAGES
\usepackage{xunicode}
\usepackage{xltxtra}
\usepackage[dvipsnames,usenames]{xcolor}
%_% XETEXPACKAGES

%_% OTHERPACKAGES
\usepackage{fontspec}
\setromanfont[Mapping=tex-text]{Linux Libertine O}
\usepackage[colorlinks=true,urlcolor=blue,linkcolor=blue]{hyperref}
%_% OTHERPACKAGES

\newcommand{\showpart}[1]{\noindent\input{parts/#1}}
\newcommand{\HyperRef}{\href{http://www.tug.org/applications/hyperref/manual.html}{hyperref}\ }
\newcommand{\FontSpec}{\href{http://tug.ctan.org/cgi-bin/ctanPackageInformation.py?id=fontspec}{fontspec}\ }

\newcommand{\red}[1]{{\addfontfeature{Colour=FF0000}#1}}

\begin{document}

\chapter{The Setup}

\section{The headers}

I use the memoir class here at 12pt default, with the option to work with much higher font sizes. The chapter style,
bianchi, is described in the
\href{ftp://tug.ctan.org/pub/tex-archive/macros/latex/contrib/memoir/memman.pdf}{documentation for the memoir class}.

The pagestyle is empty as I don't want any page numbers. The aliaspagestyle is sort of a hack so that even
chapter-opening pages don't have page numbers.

You might notice that there is a difference in the side margins between pages. This is because of the ``twoside'' option
mentioned above. This makes \LaTeX\ setup the pages for two-side printing by flipping the margins on even and odd pages.

\showpart{HEADER}

\section{Paragraph styling}

The default typesetting of paragraphs in \LaTeX\ is by indentation at the beginning of a paragraph. This has been changed
to no indent and then a bigger gap between the paragraphs, by modifying the \textbackslash{}parskip and \textbackslash{}parindent lengths.

\showpart{PARAGRAPHSTYLES}

\section{\XeTeX\ packages}

Since \XeTeX\ is being used, these three packages are normally included to be able to make the use of Unicode and some additional
functionality easier. The options passed to \href{http://www.ukern.de/tex/xcolor.html}{xcolor} to be able to use of names
instead of color codes are also needed by the \HyperRef package used below.

\showpart{XETEXPACKAGES}

\section{Font setup}

I'm using the wonderful \FontSpec package with
it's brilliant \href{http://tug.ctan.org/tex-archive/macros/xetex/latex/fontspec/fontspec.pdf}{documentation}. Also,
using the \HyperRef package to provide all the links you see in this document.

The main font being used is \href{http://linuxlibertine.sourceforge.net/}{Linux Libertine}. I'll explain what the
Mapping option is for in \autoref{fontspec-mapping}.

\showpart{OTHERPACKAGES}

\chapter{More about fonts}

\section{Ligatures}

\FontSpec understands OpenType \href{http://en.wikipedia.org/wiki/Typographic\_ligature}{ligatures}. Look at the
difference between:

%_% LIGATURES
{\newcommand{\ligs}{O\red{ft}en o\red{ffi}ce o\red{ff}er \red{fj}ord.}
{\Huge {\addfontfeature{Ligatures=NoCommon} \ligs}} \\
{\Huge {\addfontfeature{Ligatures=Rare} \ligs}}}
%_% LIGATURES

This was generated using: \showpart{LIGATURES}

\section{Mapping}\label{fontspec-mapping}

As you can see below, with the tex-text mapping all the usual quotation marks and the
\href{http://en.wikipedia.org/wiki/Dash#Em\_dash}{em-dashes} are automatically used.

%_% MAPPING
\begin{tabular}{l || l}
No tex-text mapping & {\addfontfeature{Mapping=} No---No" '' ``} \\
With tex-text mapping & {\addfontfeature{Mapping=tex-text} No---No" '' ``}
\end{tabular}
%_% MAPPING

This was generated using: \showpart{MAPPING}

\section{More font features exposed}

%_% FONTFEATURES
\newcommand{\aff}{\addfontfeature}
\begin{tabular}{l || l}
Qu has a ligature & {\Huge \red{Qu}iet \red{Q}antas} \\
Slashed zero. New style & {\Huge {\aff{Numbers=SlashedZero} 0123456789}} \\
Old style numbers & {\Huge {\aff{Numbers=OldStyle} 0123456789}} \\
Fractions & {\Huge {\aff{Fractions=On} 1/3} vs 1/3} \\
Superiors & {\Huge {\aff{VerticalPosition=Superior}1234567890 Libertine}} \\
More ligatures & {\Huge {\aff{Ligatures=Historical}„\red{st}“ „\red{ct}“}}
\end{tabular}
%_% FONTFEATURES

This was generated using: \showpart{FONTFEATURES}

\chapter{Miscallaneous}

\section{stackrel - One above another}

You can put one element above another by using stackrel as mentioned
\href{http://www.uz.ac.zw/science/maths/latex/stackrel.html}{here}:

Do this: \showpart{STACKREL}

It looks like this:
%_% STACKREL
H$_2$CO$_3$ $\stackrel{heat}{\longrightarrow}$ H$_2$O + CO$_2$
%_% STACKREL

\end{document}
